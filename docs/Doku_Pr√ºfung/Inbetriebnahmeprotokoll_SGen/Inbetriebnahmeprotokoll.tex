\documentclass[a4paper,12pt]{article}
\usepackage{amssymb} % needed for math
\usepackage{amsmath} % needed for math
\usepackage[utf8]{inputenc} % this is needed for german umlauts
\usepackage[german]{babel} % this is needed for german umlauts
\usepackage[T1]{fontenc}    % this is needed for correct output of umlauts in pdf
\usepackage[margin=2.5cm]{geometry} %layout
\usepackage{booktabs}

% this is needed for forms and links within the text
\usepackage{hyperref}

% The following is needed in order to make the code compatible
% with both latex/dvips and pdflatex.
\ifx\pdftexversion\undefined
\usepackage[dvips]{graphicx}
\else
\usepackage[pdftex]{graphicx}
\DeclareGraphicsRule{*}{mps}{*}{}
\fi

\begin{document}
\begin{center}
\section*{Inbetriebnahme-Protokoll \\ Signalgenerator}

	\begin{tabular}{|c|c|}
		\hline
		& \\
		Name des Prüfers: & \qquad \qquad \qquad \qquad \qquad \qquad \qquad \qquad \qquad \qquad \qquad \qquad \qquad \qquad \qquad \\
		& \\
		\hline
		& \\
		Prüfdatum: & \\
		& \\
		\hline
		& \\
		Seriennummer: & \\
		& \\
		\hline
	\end{tabular}
\end{center}

\section{Optische Kontrolle}

\begin{flushleft}
	\begin{tabular}{|c||l|c|c|}
		\hline
		Nr. & Prüfauftrag & Ja & Nein \\
		\hline
		1 & Sind alle Bauteile bestückt? & & \\
		\hline
		2 & Sind alle Bauteile ordnungsgemäß befestigt? & & \\
		\hline
		3 & Sind IC-Beine miteinander verbunden, & & \\
		& die nicht miteinander verbunden sein dürfen? & & \\
		\hline
		4 & der Fädeldraht zum aktivieren der USB-Schnittstelle & & \\
		& des Mikrocontrollers ist eingelötet & & \\
		\hline
		5 & Der Lötjumper zum aktivieren des LT1615 ist gesetzt & & \\
		\hline
	\end{tabular}
\end{flushleft}


\section{Elektrische Kontrolle}

\section{Funktionskontrolle}

\end{document}
