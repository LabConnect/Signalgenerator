\documentclass[a4paper,12pt]{article}
\usepackage{amssymb} % needed for math
\usepackage{amsmath} % needed for math
\usepackage[utf8]{inputenc} % this is needed for german umlauts
\usepackage[german]{babel} % this is needed for german umlauts
\usepackage[T1]{fontenc}    % this is needed for correct output of umlauts in pdf
\usepackage[margin=2.5cm]{geometry} %layout
\usepackage{booktabs}

\title{Kommunikationsprotokoll \\Signalgenerator <=> Computer}
\author{Hendrik Lüth, LabConnect}
\newcommand{\VID}{0x03EB}
\newcommand{\PID}{0x204F}


\begin{document}

\maketitle
\tableofcontents
\pagebreak

\section{Allgemeines}
In diesem Dokument wird die Datenübertragung zwischen dem Mikrocontroller des Signalgenerators und eines Computers definiert. Die Daten werden über den USB-Bus übertragen. Die USB-Spezifikationen\footnote{http://www.usb.org/developers/docs/usb20\_docs/usb\_20\_031815.zip} enthalten alle nötigen Informationen, welche für Kommunikationen über den Bus nötig sind.\\
Der Signalgenerator wird als HID (Human Interface Device) am Computer angemeldet, wodurch keine Installation von zusätzlichen Treibern nötig ist. Die Übertragung der Daten erfolgt über HID-Reports. Zum aktuellen Zeitpunk benutzt LabConnect für den Signalgenerator die VID \VID\ und die PID \PID, welche unter Linux als GenericHID von Atmel zu finden sind.\\


\section{Aufbau einer Kommunikationseinheit}
\section{Aufbau einzelner Befehle}
\subsection{Computer => Signalgenerator}
\subsection{Signalgenerator => Computer}
\section{Berechnung der Registerwerte}
\subsection{Frequenz}
\subsection{Signalform}
\subsection{Spitzenspannung}
\subsection{Offsetspannung}
\subsection{Sonstige Register}
\section{Errorcodes}

\end{document}