Ich beschäftige mich nun schon seit einer längeren Zeit mit der Thematik der Signal-Erzeugung und Verarbeitung. Mixed-Signal-Schaltungen haben mich schon immer interessiert und speziell das Verfahren der direkten digitalen Synthese hat es mir sehr angetan. Aufgrund meiner Vorarbeit zu diesem Thema wählte ich es auch zu meinem betrieblichen Auftrag. Nach diversen Absprachen zeichneten sich schnell der Rahmen des Projektes ab.\\
Mein Ziel wurde es einen Signalgenerator zu bauen, der OpenSource ist, dessen Schaltpläne und Firmware leicht zu bekommen und der auch in seinem Formfaktor eher das Gerät für die Hosentasche ist.\\
\\
Ich entschied mich auf ein externes Netzteil zu verzichten, stattdessen lieber einen Hochsetzsteller und die Spannung der USB-Schnittstelle zu verwenden. Auch wollte ich, nicht nur des Formfaktors wegen, die Bedienung und Flexibilität so einfach und hoch wie möglich machen.